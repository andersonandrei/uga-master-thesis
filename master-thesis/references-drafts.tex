% Created 2019-04-29 seg 10:55
\documentclass[a4paper,titlepage]{article}
\usepackage[utf8]{inputenc}
\usepackage[T1]{fontenc}
\usepackage{fixltx2e}
\usepackage{graphicx}
\usepackage{longtable}
\usepackage{float}
\usepackage{wrapfig}
\usepackage{rotating}
\usepackage[normalem]{ulem}
\usepackage{amsmath}
\usepackage{textcomp}
\usepackage{marvosym}
\usepackage{wasysym}
\usepackage{amssymb}
\usepackage{hyperref}
\tolerance=1000
\usepackage{pdfpages}
\usepackage{graphicx}
\usepackage[english]{babel}
\usepackage{hyperref}

\hypersetup{
  pdfkeywords={},
  pdfsubject={},
  pdfcreator={Emacs 25.2.2 (Org mode 8.2.10)}}
\begin{document}

\tableofcontents

\section{Introduction}
\label{sec-2}

[Document organization]
This document is organized as follows.
Sections:
    Smart furniture
    Environment of simulation:
        SIMGRID
        Batsim
    Job allocation problem
        Qarnot policies
    Platform 
        Real
        Simulated
    Experiments
        Results
\section{Simulation environment}
\label{sec-3}
\begin{enumerate}
\item Simgrid
\label{sec-3-1}
\begin{enumerate}
\item General


\label{sec-3-1-1}
\textbf{[Simgrid: a Toolkit for the Simulation of application Scheduling]}


Advances in hardware and software technologies have
made it possible to deploy parallel applications over in-
creasingly large sets of distributed resources. Conse-
quently, the study of scheduling algorithms for such appli-
cations has been an active area of research. Given the na-
ture of most scheduling problems one must resort to simula-
tion to effectively evaluate and compare their efficacy over
a wide range of scenarios. 

\textbf{[Scheduling Distributed Applications: the SimGrid Simulation Framework]}

\item Vesability, scalability and accuracy


\label{sec-3-1-2}
\textbf{[Accuracy Study and Improvement of Network Simulation in the SimGrid Framework]}

First, real experiments on such large-
scale distributed platforms are considerably time consuming.
Second, network devices and software have many possible
standards an implementations, leading to different behaviors
and interactions that are very difficult to control. Thus, such
systems are so complex and unstable that experiments are
not repeatable. Last, it is generally very hard to obtain such
platforms at hand to execute experiments. That is why most
research in this area resort to simulation-based studies.

\textbf{[Versatile, Scalable, and Accurate Simulation of Distributed Applications and Platforms]}

Table 1: State-of-the-art simulators from various communities and modeling approaches.

\item Energy plugins
\label{sec-3-1-3}

\textbf{[Cooling Energy Integration in SimGrid]}

Distributed computing infrastructures such as data centers
consume a considerable part of electricity consumption around the world and their extent of energy consumption is increasing day by day \footnote{DEFINITION NOT FOUND.}. From Fig. 1, it is evident that around 38\% of the total energy consumption of data centers pertains to cooling energy. 

Computation time to finish the jobs can be measured in SimGrid. Moreover, there is a module to calculate the energy consumption. This energy consumption is proportional to the workload and computation time.

As mentioned earlier, using SimGrid energy plug-in, we
can evaluate all the energy dissipations by the machines of
a distributed system. However, this energy plug-in does not
include cooling energy consumption. 

To investigate cooling energy in SimGrid, we model the
cooling energy based on CPU load and temperature difference
between environment and maximum allowable temperature of
machines. 

\item Storage simulation
\label{sec-3-1-4}

\textbf{[Adding Storage Simulation Capacities to the SimGrid Toolkit: Concepts, Models, and API]}

 Understanding the performance
of a storage subsystem thus becomes an important concern
independent of the scale and type of distributed computing
infrastructure.

 The contribution
of this work can be decomposed as follows:
• A comprehensive description of the characteristics, con-
tents, location, and access method of storage resources;
• An original API to develop SimGrid-based simulators that
manipulate storage resources and files;
• A performance analysis of various types of disks from
which we derive models used by the simulation kernel;
• A list of envisioned simulators that cover different types
of distributed infrastructures and can all be developed
based on the proposed API and models, but whose
implementation is out of the scope of this paper.

\textbf{[Predicting the Energy Consumption of MPI Applications at Scale Using a Single Node]}

Monitoring and assessing the energy efficiency of
supercomputers and data centers is crucial in order to limit
and reduce their energy consumption. Applications from the
domain of High Performance Computing (HPC), such as MPI
applications, account for a significant fraction of the overall
energy consumed by HPC centers. Simulation is a popular
approach for studying the behavior of these applications in
a variety of scenarios, and it is therefore advantageous to
be able to study their energy consumption in a cost-efficient,
controllable, and also reproducible simulation environment. 

In this work,
we aim to accurately predict the energy consumption of MPI
applications via simulation.

We propose and implement an energy model and explain
how to instantiate it by using only a single node of a cluster.
This model can later be used to predict the energy consumption
of the entire cluster.
\end{enumerate}

\item Batsim

\label{sec-3-2}
\textbf{[Batsim: a Realistic Language-Independent Resources and Jobs Management Systems Simulator]}

 Meanwhile, many scheduling
algorithms emerging from theoretical studies have not been transferred
to production tools for lack of realistic experimental validation. To tackle
these problems we propose Batsim, an extendable, language-independent
and scalable RJMS simulator. It allows researchers and engineers to test
and compare any scheduling algorithm, using a simple event-based com-
munication interface, which allows different levels of realism.

Batsim is an open source platform simulator that allows to simulate the be-
haviour of a computational platform on which a workload is executed according
to the rules of a scheduling algorithm. In order to obtain sound simulation re-
sults and to broaden the scope of the experiments that can be done thanks to
Batsim, we did not choose to build it from scratch but on top of the SimGrid
simulation framework instead.

\item Platform
\label{sec-3-3}
\begin{enumerate}
\item Real
\label{sec-3-3-1}

\item Simulated
\label{sec-3-3-2}
\end{enumerate}

\item Job Caracteristics
\label{sec-3-4}
\item Objetctives
\label{sec-3-5}
\item Simulation versions
\label{sec-3-6}
\end{enumerate}
\section{State of the art}
\label{sec-4}
\begin{enumerate}
\item Smart furniture
\label{sec-4-1}
\item Platform simulation
\label{sec-4-2}
\item Scheduling techniques
\label{sec-4-3}
\end{enumerate}
\section{Background}
\label{sec-5}
\section{Design of Experiment}
\label{sec-6}
\section{Conclusion}
\label{sec-7}
\section{Acknowledgment}
\label{sec-8}
\section{References}
\label{sec-9}
% Emacs 25.2.2 (Org mode 8.2.10)
\end{document}
